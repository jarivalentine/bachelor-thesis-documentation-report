\newpage

\section{Introduction}

\subsection{General}

In today's digital landscape real-time collaborative features are essential in modern web applications, enabling seamless communication and interaction among users. From project management tools to collaborative document editing platforms, these features enhance the way we work online. However, the successful implementation of such features relies heavily on understanding the underlying technologies, making informed choices and executing effective implementation strategies.

This research delves into the realm of real-time collaborative features within web applications. Aiming to provide a comprehensive understanding of the underlying technologies, decision-making process and implementation best practices. The scope encompasses an exploration of different protocols and technologies such as WebSocket and Server-Sent Events, as well as an analysis of relevant libraries. Additionally, the thesis addresses the challenges associated with choosing the right technology stack and effectively integrating a collaborative feature into a web application. The primary target audience comprises web developers, software engineers and technology enthusiasts seeking to enhance their understanding of real-time collaborative features.

I interned at Teamleader, a leading provider of business software solutions. This thesis emerged from Teamleader's initiative to explore alternatives to their current notification system, operated through Pusher. With a desire to transition to an in-house solution for real-time notifications, Teamleader also expressed interest in expanding their software's collaborative real-time features. Thus, this research aims to focus on the implementation of real-time collaborative functionalities within Teamleader's software suite.

\subsection{The problem}

The specific problem in this research revolves around the challenges associated with implementing real-time collaborative features in web applications. While users increasingly expect such features, the process of implementation is not straightforward. Key considerations include selecting the appropriate technology and libraries, as this choice significantly impacts performance and scalability. An additional goal of this research is to delve into the underlying concepts, which not only aids in debugging during implementation but also assists developers in making informed decisions about technology selection for their projects. The scope of the study is limited to web applications, ensuring a focused exploration.

\subsection{Research question}

The research question for this bachelor dissertation is:

\textbf{How can real-time collaborative features be effectively implemented?}

This question will guide the exploration and analysis conducted throughout the study. This primary question can further be clarified into the following sub-questions:

\begin{enumerate}
    \item How does real-time communication work in web applications?
    \item What technology is best suited for implementing real-time collaboration?
    \item What are the steps involved in implementing the chosen technology effectively?
\end{enumerate}

\subsection{Experiment}

This research is divided into three main components based on the formulated sub-questions. The approach to addressing these sub-questions and the methodologies used to obtain the answers are outlined below. The paper is conducted in English due to the international environment at the internship and a bilingual upbringing.

\begin{enumerate}
    \item Fundamentals of real-time technologies: \\ This part focuses on understanding the Fundamental workings of real-time communication in web applications, including the underlying concepts and technologies that enable it. \\ \textbf{Methodologies:} Literature review, case studies, expert interviews
    \item Choosing a library: \\ To identify the strengths and weaknesses of different libraries, a comparative analysis will be conducted. This analysis will focus on features, scalability, flexibility and ease of use. The comparison will be primarily supported by literature review. \\ \textbf{Methodologies:} Literature review, comparative analysis, benchmarking
    \item Effective implementation: \\ To answer the third sub-question, a practical approach will be taken. A prototype will be developed in which different features are implemented with the chosen technology stack. This includes setting up an environment, implementing technology-specific functionalities, and evaluating the impact on the collaborative capabilities. \\ \textbf{Methodologies:} Practical experiments, prototyping
\end{enumerate}

This structured approach will ensure that each sub-question is thoroughly investigated, providing a comprehensive understanding of real-time collaborative features and their effective implementation in web applications.
