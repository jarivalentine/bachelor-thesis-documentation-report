\newpage

\section*{Appendices}
\addcontentsline{toc}{section}{Appendices}

\subsection*{Appendix A: Case Study Figma}
\addcontentsline{toc}{subsection}{Appendix A: Case study Figma}

In today's digital landscape, real-time collaboration has become a cornerstone of productivity and innovation, revolutionising the way teams work together on projects. Figma, a leading design tool, exemplifies the power and potential of real-time communication technologies such as WebSockets. This case study delves into Figma's innovative use of WebSockets to facilitate seamless collaboration. By exploring Figma's approach to real-time communication, this study can gain valuable insights into how WebSockets can be leveraged to enhance collaboration in diverse applications and industries.

\subsubsection*{Operational Transformation (OT)}

OT is a technique for resolving a wide variety of conflicts in collaborative editing systems, including read-write conflict, merge conflicts and write-write conflict. It works by transforming the operations of one user to make them compatible with operations of another user. This allows users to work on the same document simultaneously without causing any conflict. \cite{ot}

Operational Transformation gained popularity through Google Docs \cite{figma-rtc}, enabling its real-time text editor to function seamlessly. This area is extensive, involving complex algorithms and significant research. However, to maintain focus on more fundamental collaborative features, this research will not delve into real-time text editing, despite its significance in collaborative systems like Google Docs.

\subsubsection*{Figma's context}

Figma was the first design tool to introduce real-time collaboration features. Their features eliminate the need to export, sync or email copies of files and allows more people to take part in the design process. While we use these tools everyday note that there still aren't many public case studies on these systems. But Figma decided to share how they did it in the hopes of helping others.

Figma doesn't use OT because it was unnecessarily complex for their use and they had the ability to develop their own solution. Figma's setup contains a client/server architecture with communication over WebSockets. Their servers spin a process for each document which everyone editing connects to. \cite{figma-rtc}

The server is written in Rust which dramatically improves the performance of huge design files in the tool. Rust combines high speed with low resource usage while still offering safety. Given the scale of Figma and its heavy reliance on real-time collaboration, prioritising low resource usage became important. This emerged particularly as their previous TypeScript server encountered issues. It's worth noting that while performance at this scale is critical for complex applications like Figma, simpler collaborative tools such as status boards may not require such extensive resource management. \cite{figma-rust}

Figma's flow:

\begin{enumerate}
  \item Document open: client downloads a copy of the document from stored in their document system.
  \item Document updates: to that document are synced over the WebSocket.
  \item Going offline and back online: the client applies offline edits to a fresh copy
  \item More syncing: comments, users, teams, projects stored in postgres
\end{enumerate}

\subsubsection*{Conflict-free Replicated Data Types (CRDTs)}

CRDTs are data structures that can be replicated across multiple computers in a network without coordination between them. They are designed to be conflict-free, meaning that any two replicas can be merged without conflict. In contrast to OT, CRDTs are simpler and more efficient, making them ideal for real-time collaboration systems. But that makes them less powerful and flexible than OTs. CRDTs can solve conflicts by solving e.g. a write-write conflict by letting the last write win, while OTs can solve it by transforming the operations. In Google Docs this means OTs can solve a write-write conflict by merging both writes in stead of letting the last write win. So Figma's solution is to combine multiple CRDTs into a final data structure representing the document. OTs are very complex and not needed for Figma's use case.

\subsubsection*{Figma's document structure}

A Figma document is similar to the HTML DOM. It is a tree of objects and it all starts with the root object being the document itself. Under the root there are page objects and under each page there is a hierarchy of objects representing the design (this tree is visible in the layers panel). Each object has an ID and a collection of properties with values. You can imagine this as a two-level map where the first level is the object ID and the second level is another map of properties with values.

\subsubsection*{Syncing object properties}

The server keeps track of the latest value that any client sent for a given property on a given object. When 2 clients change and unrelated property on the same page objects won't conflict. Also when 2 clients change the same property on an unrelated object there won't be a conflict. But when 2 clients change the same property on the same object there will be. Using CRDTs 'last-writer-wins' the server keeps track of the last write and uses that value. When editing a text property it is important to note that Figma also lets the last write win and won't use any OTs because Figma is not a text editor and this isn't a priority.

Another conflict occurs when a client applies it's change immediately without waiting for the server. When this happends togheter with a change from another client a flicker occurs when older server values temporarily overwrite newer ones. The solution here is to predict eventual consistency. Our change has not been acknowledged yet by the server but it is the best predication because it's the most recent change. Then we can discard all incoming changes from the server that are older than our change until the server acknowledged our change.

\subsubsection*{Syncing object creation and removal}

Creating and deleting are explicit actions and the solution is again 'last-writer-wins'. Figma does not store any deleted items or properties because they use an undo buffer: the client is responsible for keeping track of that undo buffer and this helps to document from continuing to grow. This system also relies on the clients for generating object IDs because it needs to work offline, once the user is back online al the changes are sent to the server.

\subsubsection*{Syncing object hierarchy}

Reparenting, moving an object from one parent to another, shouldn't conflict with property changes, e.g. when someone is changing a property while someone else is reparenting the object. Also 2 reparenting operations on the same object shouldn't conflict. A bad solution would be to delete the object and recreate it because the concurrent edits would be dropped. Figma's solution is to represent the parent-child relationship by storing a link to a parent as a property. This way the object identity is preserved but this also causes another problem: a cycle can occur that causes A to become a child of B and B to become a child of A. The solution here is to reject cycles in the server. To also fix this on the client side the solution is to temporarily parent these object ot each other and remove them from the tree until the server rejects and the object is reparented to the correct parent. While the object temporarily disappears it's a simple solution to a very rare problem.

Figma uses something interesting to order the children for a parent called fractional indexing. An objects position in it's parent array is stored as a fraction between 0 and 1. When inserting between 2 objects the position can be set to the average of those 2 objects. This doesn't cause ant issues when moving and adding objects.

\subsubsection*{Conclusion}

Figma's innovative use of real-time communication, specifically WebSocket, demonstrates the significant impact of such technologies on collaborative software. By leveraging CRDTs in combination with custom solutions, Figma effectively addresses common challenges in real-time collaboration, such as conflict resolution and efficient data syncing.

The decision to use Rust for their server-side implementation highlights the importance of choosing the right technology to meet performance requirements. Figma's approach to handling document structure and property syncing showcases the practical solutions to complex problems, ensuring a smooth and responsive user experience.

This case study underscores the potential of WebSocket and similar technologies to enhance real-time collaboration across various applications. It also provides valuable insights for developers seeking to implement collaborative features, emphasizing the need for careful consideration of a robust conflict resolution mechanism.
